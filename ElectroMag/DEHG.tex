\documentclass{article}
\usepackage[paperwidth=55mm,paperheight=55mm,margin=1mm]{geometry}
\usepackage{bm}
\usepackage{eucal}
\usepackage{tikz}
\usetikzlibrary{calc}
\usetikzlibrary{decorations.markings}
\pagestyle{empty}
\parindent=0pt
\begin{document}
\pgfdeclarelayer{-1}
\pgfsetlayers{-1,main}
\tikzset{
    zlevel/.style={%
        execute at begin scope={\pgfonlayer{#1}},
        execute at end scope={\endpgfonlayer}
    },
}
\centering
\begin{tikzpicture}[
        ball/.style={circle, shading=ball, ball color=black!15, minimum size=9mm},
        conline/.style={line width=#1, line cap=round},
        label/.style 2 args={
            postaction={decorate,transform shape,decoration={
                markings, mark=at position #1 with \node {\scriptsize\color{black}#2};
            }}
        },
        blue/.style={color=blue!60},
        red/.style={color=red!50},
        redl/.style={color=red!20},
    ]
    \def\conline<#1>[#2] (#3) (#4);{%
        \draw[conline=#1, #2] (#3) -- (#4);
    }
    \def\conwhiline (#1) (#2);{%
        \conline<10pt>[color=white] (#1) (#2);
    }
    \def\connectpos[#1] (#2) (#3) #4 #5;{%
        \conline<8pt>[color=black] (#2) (#3);
        \conline<7pt>[#1, label={#4}{#5}] (#2) (#3);
    }
    \def\connection[#1] (#2) (#3) #4;{%
        \connectpos[#1] (#2) (#3) 0.5 {#4};
    }
    \node (B) [ball]                     {$\bm{B}$};
    \node (D) [ball] at ($(B)+(95:4.4)$) {$\mathcal{D}$};

    \begin{scope}[zlevel=-1]
    \node (H) [ball] at ($(B)+(60:2.5)$) {$\mathcal{H}$};
    \node (E) [ball] at ($(B)+(155:3)$)  {$\bm{E}$};
    \connectpos[blue] (E)     (H.180) 0.35 {1-forms};
    \conwhiline       (B)     (D);
    \connection[red]  (E.-45) (B)          {field strength};
    \connection[redl] (B)     (H.-115)     {magnetic};
    \end{scope}

    \connection[red]  (D.-40) (H)          {excitation};
    \connectpos[blue] (D)     (B)     0.35 {2-forms};
    \connection[redl] (E.60)  (D)          {electric};
\end{tikzpicture}
\end{document}